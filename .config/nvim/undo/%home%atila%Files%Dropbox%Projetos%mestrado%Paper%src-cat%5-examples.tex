Vim�UnDo���B�����șo�$'�A�d��L���xR�Ucce�_�����2Vcce��2 \section{Results}\label{results}�In this section, the results of applying the discussed methods to the 2D Marmousi and Sigsbee 2B models are shown and briefly discussed.8\subsection{2D Marmousi example} \label{exampleMarmousi}The Marmousi model is not that deep, having only around $3.2 \mathrm{km}$; however, it has many dipping reflections, which will test the directionality of the filters. The data associated with the model have spacings of $dx = 5$ and $dz = 5$~\textrm{m}, and array dimensions of $nx = 1599$ and $nz = 639$. The shot gather is used as input for the RTM migration. A set of 320 synthetic shot gathers were used. Those shots were migrated and then remigrated, serving as the input for the filtering methods discussed throughout this paper.,The experiment had the following parameters.QOn the one hand, the space domain U-Net filter used 600 patches of size $32\times32$, with a 20-unit overlap in both directions. The validation split between training data was $0.2$, the learning rate was $lr = 0.0006$, and the number of maximum training epochs of 200. The base number of kernels in the first convolutional layer was 16.On the other, the curvelet domain U-Net filter used a learning rate of $lr = 0.002$, and respectively $nbs = 2$, $nbangles = 8$ curvelet transform scales and number of wedges in the coarsest scale ($s = 1$) with 350 patches of size $64 \times 64$ to reduce the computational cost.�They were compared with a conventional LSRTM image, after ten iterations and to the migrated image, which is equivalent to the image from the first iteration.�Both filters delivered similar results for the Marmousi model, as shown in Figures~\ref{marmComparisonUnet} and  \ref{imgdomain}. In both images, there are two dotted squares and a vertical dotted line. The squares show the area that the zoomed-in images in Figures~\ref{marmZoomComparisonUnet} and~\ref{marmZoomComparisonCurv}. The dotted line shows the offset in which the comparison trace plot is shown in Figure~\ref{comparisonCurv}6.gThe result of the space domain U-Net filter was a slight increase in the resolution of reflections at depth, close to the level of the LSRTM image. The arrow in Figure~\ref{marmFiltWindowUnet} shows a reflection that, in the migrated image in Figure~\ref{marmWindowUnet} seems blurred, which is intensified. However, it did not match the increase in illumination that the LSRTM image in Figure~\ref{marmLSRTMWindowUnet}. The same pattern repeats itself with other reflections, such that in Figures~\ref{marmWindowUnet2},~\ref{marmFiltWindowUnet2}, and~\ref{marmLSRTMWindowUnet2} the reflection indicated by the arrow shows an increase in resolution with the filter but not a comparable illumination improvement. Additionally, it can be seen that the filter took around 20 training epochs to minimize the loss function substantially, as seen in Figure~\ref{lossSpaceMarm}.DThe curvelet domain U-Net filter shows a similar increase in resolution. In Figure~\ref{marmFiltWindowCurv} the reflections in the upper-left corner are more defined, and the reflection indicated by the arrow shows a better continuity to the left compared to the migrated image in Figure~\ref{marmWindowCurv}. In Figure~\ref{marmFiltWindowCurv2} the reflections are more concise and less blurred than in Figure~\ref{marmWindowCurv2}, as can be noted by the reflection indicated by the arrow. The filter training took around seven training epochs to find a global solution minimum.�In general, both filters served their purpose of increasing resolution. However, it can be noted that although the cost of each training epoch is higher, the curvelet domain filter could find a satisfying result after less than ten iterations. Additionally, the curvelet domain filter was slightly better at improving the continuation of reflectors with a decrease in illumination over its length. From observing the trace plot comparison on Figure~\ref{tracemarm}, both filters have amplitudes very close to the RTM image, which is much higher than the LSRTM one. This is to be expected, given that the normalization model was fitted using the migrated image.\begin{figure}    \centering?    \includegraphics[width=\textwidth]{filter/vel_marmousi.png}W    \caption{Used velocity model for data modeling and migration of the Marmousi model}    \label{marmvel}\end{figure}\begin{figure}    \centering$    \begin{subfigure}{0.7\textwidth}i        \includegraphics[width=\textwidth, height=0.8\textheight/3]{filter/space_domain_marmousi-rtm.png}         \caption{Migrated image}        \label{marmUnet}    \end{subfigure}$    \begin{subfigure}{0.7\textwidth}n        \includegraphics[width=\textwidth, height=0.8\textheight/3]{filter/space_domain_marmousi-filtered.png})        \caption{Filtered migrated image}        \label{marmFiltUnet}    \end{subfigure}$    \begin{subfigure}{0.7\textwidth}k        \includegraphics[width=\textwidth, height=0.8\textheight/3]{filter/space_domain_marmousi-lsrtm.png}+        \caption{10 iterations LSRTM image}        \label{marmLSRTM}    \end{subfigure}z    \caption{Result from applying the U-Net neural network model to the original migrated image and iterative LSRTM image}    \label{marmComparisonUnet}\end{figure}\begin{figure}    \centeringP    \includegraphics[width=0.7\textwidth]{filter/space_domain_marmousi-loss.png}]    \caption{Loss over the training of the space domain U-Net filter, for the Marmousi model}    \label{lossSpaceMarm}\end{figure}\begin{figure}    \centering$    \begin{subfigure}{0.4\textwidth}X        \includegraphics[width=\textwidth]{filter/space_domain_marmousi-rtm-window0.png};        \caption{Extracted window from the  migrated image}        \label{marmWindowUnet}    \end{subfigure}    \hspace{1cm}$    \begin{subfigure}{0.4\textwidth}]        \includegraphics[width=\textwidth]{filter/space_domain_marmousi-filtered-window0.png}:        \caption{Extracted window from the filtered image}"        \label{marmFiltWindowUnet}    \end{subfigure}$    \begin{subfigure}{0.4\textwidth}Z        \includegraphics[width=\textwidth]{filter/space_domain_marmousi-lsrtm-window0.png}7        \caption{Extracted window from the LSRTM image}#        \label{marmLSRTMWindowUnet}    \end{subfigure}    \hspace{1cm}$    \begin{subfigure}{0.4\textwidth}X        \includegraphics[width=\textwidth]{filter/space_domain_marmousi-rtm-window1.png}G        %\caption{Janela quadrada extraída a partir da imagem migrada};        \caption{Extracted window from the  migrated image}        \label{marmWindowUnet2}    \end{subfigure}$    \begin{subfigure}{0.4\textwidth}]        \includegraphics[width=\textwidth]{filter/space_domain_marmousi-filtered-window1.png}H        %\caption{Janela quadrada extraída a partir da imagem filtrada}:        \caption{Extracted window from the filtered image}#        \label{marmFiltWindowUnet2}    \end{subfigure}    \hspace{1cm}$    \begin{subfigure}{0.4\textwidth}Z        \includegraphics[width=\textwidth]{filter/space_domain_marmousi-lsrtm-window1.png}H        %\caption{Janela quadrada extraída a partir da imagem filtrada}:        \caption{Extracted window from the filtered image}$        \label{marmLSRTMWindowUnet2}    \end{subfigure}L    \caption{Zoomed-in comparison between migrated and U-Net filtered image}"    \label{marmZoomComparisonUnet}\end{figure}\begin{figure}    \centering$    \begin{subfigure}{0.7\textwidth}l        \includegraphics[width=\textwidth, height=0.8\textheight/3]{filter/curvelet_domain_marmousi-rtm.png}         \caption{Migrated image}        \label{marmCurv}    \end{subfigure}$    \begin{subfigure}{0.7\textwidth}q        \includegraphics[width=\textwidth, height=0.8\textheight/3]{filter/curvelet_domain_marmousi-filtered.png})        \caption{Filtered migrated image}        \label{marmFiltCurv}    \end{subfigure}$    \begin{subfigure}{0.7\textwidth}n        \includegraphics[width=\textwidth, height=0.8\textheight/3]{filter/curvelet_domain_marmousi-lsrtm.png}+        \caption{10 iterations LSRTM image}        \label{marmLSRTMCurv}    \end{subfigure}�    \caption{Result from applying the curvelet domain U-Net neural network model to the original migrated image and the iterative LSRTM image}    \label{comparisonCurv}\end{figure}\begin{figure}    \centeringS    \includegraphics[width=0.7\textwidth]{filter/curvelet_domain_marmousi-loss.png}`    \caption{Loss over the training of the curvelet domain U-Net filter, for the Marmousi model}    \label{lossCurvMarm}\end{figure}\begin{figure}    \centering$    \begin{subfigure}{0.4\textwidth}[        \includegraphics[width=\textwidth]{filter/curvelet_domain_marmousi-rtm-window0.png};        \caption{Extracted window from the  migrated image}        \label{marmWindowCurv}    \end{subfigure}    \hspace{1cm}$    \begin{subfigure}{0.4\textwidth}`        \includegraphics[width=\textwidth]{filter/curvelet_domain_marmousi-filtered-window0.png}:        \caption{Extracted window from the filtered image}"        \label{marmFiltWindowCurv}    \end{subfigure}$    \begin{subfigure}{0.4\textwidth}]        \includegraphics[width=\textwidth]{filter/curvelet_domain_marmousi-lsrtm-window0.png}:        \caption{Extracted window from the filtered image}#        \label{marmLSRTMWindowCurv}    \end{subfigure}    \hspace{1cm}$    \begin{subfigure}{0.4\textwidth}[        \includegraphics[width=\textwidth]{filter/curvelet_domain_marmousi-rtm-window1.png};        \caption{Extracted window from the  migrated image}        \label{marmWindowCurv2}    \end{subfigure}$    \begin{subfigure}{0.4\textwidth}`        \includegraphics[width=\textwidth]{filter/curvelet_domain_marmousi-filtered-window1.png}:        \caption{Extracted window from the filtered image}#        \label{marmFiltWindowCurv2}    \end{subfigure}    \hspace{1cm}$    \begin{subfigure}{0.4\textwidth}]        \includegraphics[width=\textwidth]{filter/curvelet_domain_marmousi-lsrtm-window1.png}:        \caption{Extracted window from the filtered image}$        \label{marmLSRTMWindowCurv2}    \end{subfigure}`    \caption{Zoomed-in comparison between migrated and the curvelet domain U-Net filtered image}"    \label{marmZoomComparisonCurv}\end{figure}\begin{figure}    \centeringF    \includegraphics[height=0.5\textheight]{filter/marmousi-trace.png}V    \caption{Comparisson of the two methods at a trace in the 4500~\textrm{km} offset}    \label{tracemarm}\end{figure}5\subsection{Sigsbee2A example} \label{exampleSigsbee}�The Sigsbee model represents a salt dome located offshore, which means that the objective is to represent the reflections below it better since there is less illumination in this region. The data associated with the model have spacings of $dx = 15.24$ and $dz = 7.62$~\textrm{m}, and dimensions of $nx = 1201$ and $nz = 1600$. It is a large model with a depth of 12184~\textrm{km}. A shot gather composed of 465 synthetic shots was created.FAs in the last section, the following experiment parameters were used.0For the space domain U-Net filter, 1000 patches of dimensions $32\times32$, with an overlap of 10 units in both directions. The validation split was $0.2$, with a learning rate of $lr = 0.00005$ and a maximum 200 training epochs. The base number of kernels in the first convolutional layer was set to 32.�For the curvelet domain U-Net model, the learning rate was the same, the number of scales in the curvelet transform was $nbs = 2$, and the number of wedges in the coarsest scale ($s = 1$) was $nbangles = 8$. The number of patches was the same but with a size of $64 \times 64$, with maximum number of 50 epochs. The base number of kernels in the first convolutional layer was set to 16.vGiven the Sisgbee 2A model size, the comparison with the LSRTM image was not made in contrast with the Marmousi model.�In Figure~\ref{sigsbeeComparisonUnet} is visible that there is a degradation in image quality in the filtered image in Figure \ref{imgdomain}. In the valley-like region on the top of the salt dome shown in Figure~\ref{sigsbeeFiltWindowUnet2} that there is considerable distortion of the reflectors around this area. In depth is possible to observe that reflections do not appear as continuous, making the faults clearly shown in original migrated image shown in Figure~\ref{sigsbeeWindowUnet} are blurry in the filtered image shown in Figure~\ref{sigsbeeFiltWindowUnet}. It can also be noted that the loss was high even with 80 training epochs.�On the other hand, the curvelet domain U-Net filter shows an increase in resolution akin to its performance on the Marmousi model, as seen in Figure~\ref{sigsbeeComparisonUnet}. Taking a closer look at the valley-like region above the dome in Figures~\ref{sigsbeeFiltWindowCurv2}, there are fewer distortions except for the reflector that delineates the dome. In Figure~\ref{sigsbeeFiltWindowUnet}, the filter managed to increase the continuation of reflectors with low illumination, as it can be seen on both the one pointed by the arrow and in the horizontal reflector at the bottom. The filter took eight training epochs to achieve a better result than the one given by the space domain filter, reaching a minimum loss function curve in only four iterations.�In general, only the curvelet domain filter could find a good enough result, which increased resolution. The cost is a much higher computational cost in training and inference but a smaller number of epochs to find a good result. It can be seen in the trace plot comparison in Figure~\ref{tracesigsbee} that the space domain filtered image shows a certain amount of high-frequency noise when compared with the one in the curvelet domain.\begin{figure}    \centering@    \includegraphics[width=\textwidth]{filter/vel_sigsbee2A.png}Y    \caption{Used velocity model for data modeling and migration of the Sigsbee 2A model}    \label{sigsbeevel}\end{figure}\begin{figure}    \centering$    \begin{subfigure}{0.7\textwidth}Q        \includegraphics[width=\textwidth]{filter/space_domain_sigsbee2A-rtm.png}         \caption{Migrated image}        \label{sigsbeeUnet}    \end{subfigure}    \hspace{1cm}$    \begin{subfigure}{0.7\textwidth}V        \includegraphics[width=\textwidth]{filter/space_domain_sigsbee2A-filtered.png})        \caption{Filtered migrated image}        \label{sigsbeeFiltUnet}    \end{subfigure}x    \caption{Result from applying the U-Net neural network model to the original migrated image of the Sigsbee 2A model}!    \label{sigsbeeComparisonUnet}\end{figure}\begin{figure}    \centeringQ    \includegraphics[width=0.7\textwidth]{filter/space_domain_sigsbee2A-loss.png}^    \caption{Loss over the training of the space domain U-Net filter, for the Sigsbee2A model}    \label{lossSpaceSigs}\end{figure}\begin{figure}    \centering$    \begin{subfigure}{0.4\textwidth}Y        \includegraphics[width=\textwidth]{filter/space_domain_sigsbee2A-rtm-window0.png}:        \caption{Extracted window from the migrated image}!        \label{sigsbeeWindowUnet}    \end{subfigure}    \hspace{1cm}$    \begin{subfigure}{0.4\textwidth}^        \includegraphics[width=\textwidth]{filter/space_domain_sigsbee2A-filtered-window0.png}:        \caption{Extracted window from the filtered image}%        \label{sigsbeeFiltWindowUnet}    \end{subfigure}$    \begin{subfigure}{0.4\textwidth}Y        \includegraphics[width=\textwidth]{filter/space_domain_sigsbee2A-rtm-window1.png}:        \caption{Extracted window from the migrated image}"        \label{sigsbeeWindowUnet2}    \end{subfigure}    \hspace{1cm}$    \begin{subfigure}{0.4\textwidth}^        \includegraphics[width=\textwidth]{filter/space_domain_sigsbee2A-filtered-window1.png}:        \caption{Extracted window from the filtered image}&        \label{sigsbeeFiltWindowUnet2}    \end{subfigure}L    \caption{Zoomed-in comparison between migrated and U-Net filtered image}%    \label{sigsbeeZoomComparisonUnet}\end{figure}\begin{figure}    \centering$    \begin{subfigure}{0.7\textwidth}T        \includegraphics[width=\textwidth]{filter/curvelet_domain_sigsbee2A-rtm.png}         \caption{Migrated image}        \label{sigsbeeCurv}    \end{subfigure}    \hspace{1cm}$    \begin{subfigure}{0.7\textwidth}Y        \includegraphics[width=\textwidth]{filter/curvelet_domain_sigsbee2A-filtered.png})        \caption{Filtered migrated image}        \label{sigsbeeFiltCurv}    \end{subfigure}�    \caption{Result from applying the curvelet domain U-Net neural network model to the original migrated image of the Sigsbee 2A model}!    \label{sigsbeeComparisonCurv}\end{figure}\begin{figure}    \centeringT    \includegraphics[width=0.7\textwidth]{filter/curvelet_domain_sigsbee2A-loss.png}a    \caption{Loss over the training of the curvelet domain U-Net filter, for the Sigsbee2A model}    \label{lossCurvSigs}\end{figure}\begin{figure}    \centering$    \begin{subfigure}{0.4\textwidth}\        \includegraphics[width=\textwidth]{filter/curvelet_domain_sigsbee2A-rtm-window0.png};        \caption{Extracted window from the  migrated image}!        \label{sigsbeeWindowCurv}    \end{subfigure}    \hspace{1cm}$    \begin{subfigure}{0.4\textwidth}a        \includegraphics[width=\textwidth]{filter/curvelet_domain_sigsbee2A-filtered-window0.png}:        \caption{Extracted window from the filtered image}%        \label{sigsbeeFiltWindowCurv}    \end{subfigure}$    \begin{subfigure}{0.4\textwidth}\        \includegraphics[width=\textwidth]{filter/curvelet_domain_sigsbee2A-rtm-window1.png};        \caption{Extracted window from the  migrated image}"        \label{sigsbeeWindowCurv2}    \end{subfigure}    \hspace{1cm}$    \begin{subfigure}{0.4\textwidth}a        \includegraphics[width=\textwidth]{filter/curvelet_domain_sigsbee2A-filtered-window1.png}:        \caption{Extracted window from the filtered image}&        \label{sigsbeeFiltWindowCurv2}    \end{subfigure}`    \caption{Zoomed-in comparison between migrated and the curvelet domain U-Net filtered image}%    \label{sigsbeeZoomComparisonCurv}\end{figure}\begin{figure}    \centeringG    \includegraphics[height=0.7\textheight]{filter/sigsbee2A-trace.png}Z    \caption{Comparisson of the two methods at a trace in the 13319.76~\textrm{km} offset}    \label{tracesigsbee}\end{figure}5��2�G5��