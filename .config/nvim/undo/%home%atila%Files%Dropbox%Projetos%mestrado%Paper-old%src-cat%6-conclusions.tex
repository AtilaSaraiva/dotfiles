Vim�UnDo���B�����șo�$'�A�d��L���xR�UccfY_�����VccfX�\section{Conclusions}% Intro,In summary, we have presented a comparison between two neural network based filters that aim to achieve a deblurring effect when applied to the migrated image, one based on the work of \citet{Avila2021}, the other proposed by us as an adaptation of the latter to a curvelet transform domain scenario.zBoth presented interesting results, clearly performing well in the Sigsbee2B model, and not so well in the Marmousi model.�%Por um lado, o filtro U-Net no domínio espacial ofereceu uma performance mais robusta, demonstrando resultados similares tanto no modelo Marmousi quanto no Sigsbee2B. No entanto, o mesmo não foi muito capaz de melhorar a continuidade de refletores com pouquíssima iluminação, como os da lateral esquerda da Figura~\ref{marmUnet}, ou os logo abaixo do domo de sal do modelo Sigsbee2B na parte à esquerda na Figura~\ref{sigsbeeWindowUnet}.�On one hand, the spatial domain U-Net filter offered a more consistent performance, demonstrating similar results both in the Marmousi and Sigsbee2B model. However, the same was not much capable of improving the continuity of reflectors that lacked illumination because of the modeling geometry, such as those at the left of the Figure~\ref{marmUnet}, or those below the salt dome of the Sigsbee2B model, in the left part of the Figure~\ref{sigsbeeWindowUnet}.�%Por outro lado, o filtro no domínio curvelet ofereceu resultados ambíguos, sendo muito efetivo no modelo Sigsbee2B, ao custo de introduzir ruído no modelo Marmousi. Apesar de seus resultados não serem tão robustos quanto à sua contraparte no domínio do espaço, ele demonstrou ser capaz de além de melhorar a iluminação dos refletores em profundidade, de também melhorar a continuidade dos mesmos em regiões de baixa iluminação, como pode ser visto na Figura~\ref{sigsbeeFiltWindowCurv}.�On the other hand, the curvelet domain filter offered ambiguous results, being very effective in the Sigsbee2B model, at the cost of introducing some noise in the Marmousi model. Although their results are not as robust as their spatial domain counterpart, it has been shown to be able to improve illumination of reflections at depth, on top of improving their continuity in poorly illuminated regions, as can be seen in Figure~\ref{sigsbeeFiltWindowCurv}.�%On one hand, the U-Net in the spatial domain filter had better handling of it, since it used several patches with overlap, on top of having $L_1$ regularization and the last layers and some dropout layers in the encoding phase. This allowed it to not overfit too much making decisions of where to boost amplitude using the memory of all the patches in the whole image. However, this led to reflections to lose continuity.�One possible reason for this intolerance for noise is that the curvelet domain filter trains individual networks for each of its features, making it incapable to make informed decisions based on the inverse transform of the output by design.This makes different wedges at the same scale introduce artifacts constructively in the same location in the spatial domain, usually at corners since they have features of many different angles. This was the main reason why we used a low number of angles in the Sigsbee model.4If there was an implementation of the transform compatible with Tensorflow (or others) gradient implementation, it would be possible to calculate the cost function in the spatial domain by applying the inverse transform to the output of the or network, while still being able to perform auto differentiation.In fact, the work from \citet{Sanavi2021}, as mentioned in the introduction, also performs a curvelet domain filter with a cost function in the spatial domain, and instead of using a neural network, they used an optimization algorithm from the quasi-Newton family.�One way we found to reach a similar result and handle the noise was to enforce $L_1$ regularization since the curvelet domain is extremely sparse, so the output of the neural network is also supposed to be sparse. This greatly reduced the noise.ZBy using a spatial domain cost function, we could most likely achieve much better results.�Nevertheless, training and applying the U-Net model in the curvelet domain managed to deliver incredible results in the Sigsbee model, with the proper adaptations.�Several experiment hyperparameters were tested, and the ones selected were, by try and error, the best. The patch size in particular was one parameter that did not show much of an effect with an increase in size after a certain threshold. The employed methodology focused on trying to test the boundaries of deblurring filters and did not aim into doing an in-depth analysis of the optimal parameters for each method. As more works explore the concept of using a different domain for training neural networks, then a more focused evaluation of those parameters would make much more sense. In the end, it is important to note that the methodology employed was not conceptualized to guarantee true amplitude. The balanced illumination from the filter application might visually improve the interpretation of geological features, but further research is needed if one wishes to apply them to an amplitude versus offset study.:This work aims to start a discussion of using different domains for the input of the neural network in seismic imaging and correlated areas. The progress made here is only but a step in the right direction. These domains do not necessarily need to be related to decedents of the Fourier transform but could be represented by another neural network made for reparametrization. Solving the problem of calculating the cost function in the spatial domain is the most obvious one but there are many more improvements that can be done. First, by transforming the data to a sparser domain, unless there is a sparse array implementation to back it up, it will only consume more RAM memory. A better implementation of the curvelet transform that takes that into account given a certain coefficient threshold, would improve the performance of the training in general. Secondly, testing similar approaches with real data would shed a light on what types of geological features would be better illuminated. Thirdly, using state-of-the-art migration techniques, with exotic image conditions \citep[e.g][]{Moradpouri2017} and elastic wave extrapolation. Fourth, incorporating some physics-informed neural network \citep{Raissi2019} to make the model robust to many different data, which would solve the limitation of having to retrain the network on every input model change. Last but not least, it is of paramount importance to test the discussed filters as a pre-conditioner in a traditional iterative LSRTM scheme, in order to test its effectiveness in reducing the number of iterations until convergence.5���5��