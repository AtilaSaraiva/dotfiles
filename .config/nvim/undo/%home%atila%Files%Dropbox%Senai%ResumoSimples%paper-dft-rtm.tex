Vim�UnDo��K����~�~�[>�F�«!ԔI�)uo��hV"u\.    &b��_�'����&)))V)b���%+*�    The state-of-the-art family of implementations of the acoustic reverse time migration for GPU generally involves some effective border technique, which saves only the border from the direct wavefield, reconstructing it in parallel with the adjoint wavefield calculation. Consequently, this technique performs at least three full seismic wavefield modeling, not counting the ones necessary for the snapshot reconstruction. We propose a multi GPU implementation of an already established technique in which the calculation of both the direct and adjoint wavefields are calculated simultaneously while the imaging condition correlates the snapshots in the frequency domain.�&'*�&'�This correlation is done by applying a kernel from the Discrete Fourier Transform (DFT) along the time axis for both fields, moving them to the frequency domain. As a result, the snapshots from different points in time are correlated with their frequency decomposition. This is especially interesting for memory-constrained environments, where the ability to split the direct and adjoint field computations to a pair of GPUs is enough to solve the memory bottleneck. The latter, for instance, will hardly have enough VRAM to handle large 3D models, while it might be able to hold snapshots for the wave propagation at least. Another use case for such a technique is for elastic, viscoacoustic, and viscoelastic RTM, given that the effective border technique can not be applied in those cases. For such cases, some checkpointing algorithm is generally employed, which for large 3D models can represent a massive increase in memory consumption.hThe physical coherence of the DFT-RTM is related to how well the DFT can correlate the time steps to a certain number of frequencies. Although, it passes the inner product test, the resulting image differs slightly from the conventional RTM. Nevertheless, it serves its imaging purpose and can be used for an early stage of the seismic  interpretation process.$The algorithm was implemented using the OpenACC+OpenMP directives for Fortran. A simple but fast language like Fortran and OpenACC directives allowed us to investigate how this technique can better perform in a multi-GPU environment. The Nvidia HPC SDK suite, which provides the nvfortran compiler, was used. The code is available at GitHub, at the following reposotory: github.com/AtilaSaraiva/Multigpu-DFT-RTM. The results were generated using two Nvidia Tesla V100 with 32 Gbs of VRAM, in a node from the Senai Cimatec Supercomputer named Ogbon.5��&�=�%�:��%�:�5��