Vim�UnDo���B�����șo�$'�A�d��L���xR�Uccf)_�����"Vccf(�"\section{Introduction}�With deeper and deeper targets in seismic exploration, it becomes harder to image reflections at such depths. The obvious solution is to be creative in the acquisition phase to have bigger offsets, 3C receivers, 3D geometries, and so on; however, that only increases the cost. For that reason, it is much cheaper to improve the imaging techniques being employed. Being able to resolve deeper reflections on already acquired RTM data with fast techniques then becomes that much more attractive.�Least-squares reverse time migration (LSRTM) is a usually employed inversion technique since it solves some of the problems of traditional migration methods without having to resort to more sophisticated, though costly, methods such as multi-parameter FWI, for subsurface model inversion. The result of it, when compared to the migrated image, is an image with a more balanced illumination at deeper regions, more concise reflections, and less blur.% cite something hereNevertheless, the traditional way to do an LSM has been by using iterative methods such as \textit{steepest descent} and \textit{conjugate gradient}, taking several iterations to reach the final target. This increases its cost to around 10 times the cost of the conventional pre-stack migration. As a consequence, there have been two main lines of research to alleviate this cost. The first consists of finding preconditioners to ensure fast convergence \citep[e.g.][]{Dutta2015,Dutta2016,Dutta2017,Liu2017a,Sun_2015}.�The second, which this work fall into, is to find deblurring filters that reap some of the benefits from the LSM, while only being applied to the data in one iteration once calculated.�There have been multiple research proposing different ways to reach such a filter. The general idea proposed by \citet{Guitton2004}, and also hinted by \citet{Claerbout1992}, is to approximate the inverse of the Hessian ($\bfH^{-1}$) by calculating a matching filter between the remigrated image\footnote{The remigrated image is the result of migrating the demigrated shot gather} and the migrated image, to apply it on the migrated image afterward. \citet{Hu2001} proposed mostly the same methodology but assumed a horizontally invariant velocity model. \citet{Aoki2009} suggested using reference models to calculate the filter. \citet{Wang2016} calculated the filter in the curvelet domain.There is one recent work in particular \citep{Sanavi2021} that uses Limited-Memory BFGS \citep{Nocedal2006} as the optimization method of choice to calculate a similar filter, however, the cost function is calculated in the spatial domain by applying the inverse curvelet transform.�\citet{Liu2018} used a simple Wiener deconvolution filter to apply to the observed data, using the Fourier transform to calculate and apply it trace by trace. In the next year, \citet{Liu2019} used Gabor deconvolution to achieve a similar filter, but now non-stationary.  More recently, \citet{Avila2021} designed a U-Net neural network to act as a matching filter. U-Net is a type of convolutional neural network (CNN) with encoding and decoding with skip connections built-in.�It is noticeable that most of the recent works employed recently used either a transformation from the Fourier family or a neural network.�All of the mentioned methods are very similar to the ones applied in other imaging problems such as image restoration, inpainting, denoising, image segmentation. The general idea is always to model a signal $\bfy$ as the output of a system $\bfT$ (inverse of the Hessian), in which the input is denoted by $\bfx$ \citep{Lucas2018}. The difference is that the system $\bfH^{-1}$ was replaced by a filter which is inferred from the output $\bfy$ (remigrated image) and input $\bfx$ (migrated image).�On the other hand, the U-Net approach has many parameters inside its hidden layers, and its encoding and decoding nature should mean it has a similar level of local resolution. However, there is still much we don't know about neural networks, so it is difficult to design a formal mathematical model to express what the trained CNN does. So there have been many works that aim to better understand what they do, like \citet{Papyan2016}.
In this work, we try to investigate if we could expand on the approach used by \citet{Avila2021}, by implementing a matching filter on the curvelet domain. The idea is to use a similar U-Net neural network as the filter but applied in the curvelet domain coefficients of the remigrated and migrated image. The resulting would be a mesh between the \citet{Wang2016} and \citet{Avila2021} methods; approaching the limit of what deblurring filter-based LSRTM can provide. This is the case because curvelet transform expresses the data based on the scale, direction, and location of its features, being the state-of-the-art data transformation technique. This multi-parametrization would allow for the neural network to match for the same aspects of the same features. However, it feels like there is a limit to how much you can achieve by making use of mathematical transformations as preprocessors. In this work, we test this limit by comparing this new curvelet domain neural network filter, with the one proposed by \citet{Avila2021}.8The idea of training a neural network in a different domain is not new, although not so common. The work from \citet{Pratt2017} was the one that proposed training CNNs with input data in the Fourier domain to decrease the computational burden of convolution operations by performing them in the frequency domain.�However, for the matching filter to work, it has to parametrize the image effectively. If the filter only uses a simple 2D Fourier transform, local variations in frequency will not be taken into account, which will result in coefficients of the same frequency being matched while they might be from different parts of the image. Not only that but also coefficients are not separated by angle. The curvelet transform solves all of those problems while still having an explicit mathematical model.�Nevertheless, there have been very few works on using neural networks in the curvelet domain, whereas none are related to seismic processing research. \citet{Saxena2015} uses the curvelet transform to extract features from fingerprint data, and used it as input to a simple feed-forward network classifier. \citet{Gupta2014} uses a similar approach, although to detect lung cancer CT scan images.zThe purpose of this work is to use a similar idea but to a regression problem, the LSRTM deblurring filter estimation one.�We start by first describing the methodology in section~\ref{methodology} employed, with a brief introduction to LSRTM followed by a brief description of the matching filters from the two approaches. In section~\ref{curveletSection}, the curvelet transform is introduced, in order to make sense out of the filter that used it. In section~\ref{cnn}, the same is done for convolutional neural networks, while also explaining the idea behind U-Net and what adaptations have to be done for it to work with LSRTM. In sections~\ref{exampleMarmousi} and~\ref{exampleSigsbee}, the filters are applied to the Marmousi and Sigsbee model respectively, followed by a conclusion.=%In this work, the methodologies proposed in both \citet{Avila2021} and \citet{Wang2016} will be explained and compared in terms of image quality and number of artifacts. The reasoning behind it is that those two are the argueably the most advanced techiniques employed in the field of single iteration LSRTM filters.�%All of mentioned methods are very similar to the ones applied in other imaging problems such as image restoration, inpainting, denoising, image segmentation. The general idea is always to model an signal $\bfy$ as the output of a system $\bfT$, in which the input is denoted by $\bfx$ \citep{Lucas2018}. The difference is that the system $\bfH^{-1}$ was replaced by a filter which is infered from the output $\bfy$ (remigrated image) and input $\bfx$ (migrated image).�%On the other, the Unet approach has many parameters inside its hidden layers, and its encoding and decoding nature should mean it has a similar level of local resolution. However, there still much we don't know about neural networks, so it is dificult to design a formal mathematical model to express what the trained CNN does. So there have been many works that aim to better understand what they do, like \citet{Papyan2016}.�%From the comparison between the two methodologies, we aim to better understand how the neural network positions itself compared to the one of the most advanced transformation based technique to this moment.�%We start by first describing the methodology in section \ref{methodology} employed, with a brief introduction to LSRTM followed by a brief description of the matching filters from the two approaches. In section \ref{curveletSection}, the curvelet transform is introduced, in order to make sense out of the filter that used it. In section \ref{cnn}, the same is done for convolutional neural networks, while also explaining the idea behind Unet and what adaptations have to be done for it to work with LSRTM. In sections \ref{exampleMarmousi} and \ref{exampleSigsbee}, the filters are applied to the Marmousi and Sigsbee model respectively, followed by a conclusion./%, like unbalaced ilumination on deeper regions5��"�$5��