Vim�UnDo���B�����șo�$'�A�d��L���xR�Ucce�_�����Vcce��\section{Discussion}The image improvement observed in both filters, except in the case of the space domain filter in the Sigsbee 2A model, is visible. Nevertheless, the illumination and amplitude scale is still different compared to the LSRTM image, as seen in Figures A0 and A1.zThe conventional iterative LSRTM, from the mathematical standpoint, is better than any other single iteration deblurring scheme that tries to mimic it. This stems from the fact that it sits in the mathematical foundation of linear inversion techniques that have decades of research, on top of accurately describing how incrementally improve the solution at each iteration using gradient-based algorithms. All the numerical techniques that try to speed up the convergence using preconditioning or a priori information do not change this foundation, improve upon it, and reap its benefits. Those techniques use cost functions similar to\begin{equation}6	E(\mathbf{m}) = \| \mathbf{Lm} - \mathbf{d} \|_2^2.$$\end{equation}On the other hand, techniques that stray away from this cost function to either try to approximate the inverse of the Hessian $\mathbf{H}^{-1}$, or do some other trick to avoid solving the cost function above iteratively, are in essence trying to approximate a function (or operator) instead of the model $\mathbf{m}$ itself. This is bound to have poorer results because trying to approximate a multi-dimensional tensor like $\mathbf{H}^{-1}$ is very unstable and model-dependent. This work falls into this category.�One of the possible reasons for the noise in the space domain filter applied to the Sigsbee 2A model is that the model mainly consists of parallel layers, given that the salt dome occupies only a tiny portion of the model, which causes an imbalance in the number of patches with dipping reflections. The balancing preprocessing workflow is commonplace in neural network training, in which the training samples are categorized by their characteristics. Some technique is then used to balance the number of samples in each category. This is done to avoid making the neural network stuck in a local minimum without reaching the global minimum. The effects of this lack of balance can be seen in Figure A2, in which reflections around the 5000~\textrm{m}, which are all parallel horizontal layers, have a better illumination than the ones around the offset 10000~\textrm{m}, which are mostly dipping reflections.�The curvelet transform acts as a balancing operator because it separates a model for each curvelet feature, which is associated with different dips and scales. This explains why the curvelet domain filter worked well with this model.�It is important to note that in all three papers that were previously mentioned that implemented similar neural network single iteration filters [0], none tested their filters in the Sigsbee 2A model.�Several experiment hyperparameters were tested, and the ones selected were, by try and error, the best. The patch size, in particular, was one parameter that did not show much effect with an increase in size after a certain threshold. The employed methodology focused on trying to test the boundaries of deblurring filters and did not aim to do an in-depth analysis of the optimal parameters for each method. As more works explore using a different domain as a latent space for training neural networks, a more focused evaluation of those parameters would make much more sense. Ultimately, it is essential to note that the methodology employed was not conceptualized to guarantee true amplitude. The balanced illumination from the filter application might visually improve the interpretation of geological features, but further research is needed if one wishes to apply them to an amplitude versus offset study.\section{Conclusions}*In summary, we have presented a comparison between two neural network-based filters that aim to achieve a deblurring effect when applied to the migrated image, one based on the work of~\citep{Dai2013}, the other proposed by us as an adaptation of the latter to a curvelet transform domain scenario.wBoth presented exciting results, clearly performing well in the Marmousi model and not so well in the Sigsbee 2A model.�On the one hand, the curvelet domain U-Net filter offered a more consistent performance, demonstrating similar results in the Marmousi and Sigsbee 2A models. However, it came with a significant increase in computational cost, which scaled almost linearly with the number of curvelet features used. Still, the results were consistent, and the hyperparameters for the model were easier to tune, most likely due to the curvelet transform acting as a balancing operator.In contrast, the space domain U-Net filter was not performant enough in the Sigsbee 2A model, although it performed pretty well in the Marmousi. Both filters achieved a deblurring effect in some capacity but could not match the better illumination of the conventional LSRTM image.4This work aims to start a discussion of using different domains for the input of the neural network in seismic imaging and correlated areas. The progress made here is only but a step in the right direction. These domains do not necessarily need to be related to decedents of the Fourier transform but could be represented by another neural network made for reparametrization. Nevertheless, from the observations done in this paper, a natural progression for single-step deblurring techniques is to implement them as preconditioners. Another promising option is optimizing the weights iteratively by demigrating the result of the neural network and calculating its residue with the observed data. Doing the latter would allow the filter to progressively approximate the Hessian inverse instead of with only one data pair.5��n5��