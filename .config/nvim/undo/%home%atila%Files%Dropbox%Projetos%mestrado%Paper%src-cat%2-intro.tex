Vim�UnDo���B�����șo�$'�A�d��L���xR�Ucce{_�����Vccez�\section{Introduction}�With deeper and deeper targets in seismic exploration, it becomes harder to image reflections at such depths. The obvious solution is to be creative in the acquisition phase to have bigger offsets, 3C receivers, and 3D geometries; however, that only increases the cost. For that reason, improving imaging techniques in the processing phase is much cheaper. Being able to resolve more profound reflections on already acquired RTM data with fast techniques becomes much more attractive.�Least-squares reverse time migration (LSRTM) is a usually employed inversion technique since it solves some of the problems of traditional migration methods without resorting to more sophisticated, though costly, methods such as multi-parameter FWI, for subsurface model inversion. Compared to the migrated image, it has a more balanced illumination at deeper regions, more concise reflections, and less blur.Nevertheless, the traditional way to do an LSM has been by using iterative methods such as \textit{steepest descent} and \textit{conjugate gradient}, taking several iterations to reach the final target. This iterative scheme increases its cost to several times the cost of the conventional pre-stack migration. Consequently, there have been two main lines of research to alleviate this cost. The first consists of finding preconditioners to ensure fast convergence \citep[e.g.][]{Dutta2015,Dutta2016,Dutta2017,Liu2017a,Sun_2015}.�The second, which this work falls into, is to find deblurring filters that reap some of the benefits from the LSM while only being applied to the data in one iteration once calculated.�There has been considerable research proposing different ways to reach such a filter. The general idea proposed by \citet{Guitton2004}, and also hinted by \citet{Claerbout1992}, is to approximate the inverse of the Hessian ($\bfH^{-1}$) by calculating a matching filter between the remigrated image\footnote{The remigrated image is the result of migrating the demigrated shot gather} and the migrated image, to apply it on the migrated image afterward. \citet{Hu2001} proposed mostly the same methodology but assumed a horizontally invariant velocity model. \citet{Aoki2009} suggested using reference models to calculate the filter. \citet{Wang2016} calculated the filter in the curvelet domain.One recent work in particular \citep{Sanavi2021} uses Limited-Memory BFGS \citep{Nocedal2006} as the optimization method of choice to calculate a similar filter. However, the spatial domain calculates the cost function by applying the inverse curvelet transform.|\citet{Liu2018} used a simple Wiener deconvolution filter to apply to the observed data, using the Fourier transform to calculate and apply it trace by trace. In the following year, \citet{Liu2019} used Gabor deconvolution to achieve a similar filter, but now non-stationary. Recently, \citet{Kaur2020} proposed a generative adversarial network (GAN) as a single iteration filter.\citet{Avila2021} designed a U-Net neural network to act as a matching filter. U-Net is a type of convolutional neural network (CNN) with encoding and decoding with skip connections built-in. \citet{Torres2021} proposes a similar filter as one of its employed methodologies.�It is noticeable that most recent works employed recently used either a transformation from the Fourier family or a neural network.�All of the mentioned methods are very similar to the ones applied in other imaging problems, such as image restoration, inpainting, denoising, and image segmentation. The general idea is always to model a signal $\bfy$ as the output of a system $\bfT$ (inverse of the Hessian), in which the input is denoted by $\bfx$ \citep{Lucas2018}. The difference is that the system $\bfH^{-1}$ was replaced by a filter which is inferred from the output $\bfy$ (remigrated image) and input $\bfx$ (migrated image).�On the other hand, the U-Net approach has many parameters inside its hidden layers. Its encoding and decoding nature should mean it has a similar level of local resolution. However, we still do not know much about neural networks, so designing a formal mathematical model to express what the trained CNN does is challenging. Thus, many works have aimed to better understand how they work, like \citet{Papyan2016}.�In this work, we investigate if we could expand on the approach used by \citet{Avila2021} by implementing a neural network matching filter on the curvelet domain. The idea is to use a similar U-Net neural network as the filter but applied in the curvelet domain coefficients of the remigrated and migrated images. The resulting filter would be a mesh between the \citet{Wang2016} and \citet{Avila2021} methods, approaching the limit of what deblurring filter-based LSRTM can provide. The state-of-the-art data transformation technique, the curvelet transform, expresses the data based on its features' scale, direction, and location. This multi-parametrization would allow the neural network to match the same aspects of the same features. However, it feels like there is a limit to how much can be achieved by using mathematical transformations as preprocessors. In this work, we test this limit by comparing this new curvelet domain neural network filter with the one proposed by \citet{Avila2021}.,Training a neural network in a different domain is not new, although not so common. The work from \citet{Pratt2017} was the one that proposed training CNNs with input data in the Fourier domain to decrease the computational burden of convolution operations by performing them in the frequency domain.�However, for the matching filter to work, it has to parametrize the image effectively. Suppose the filter only uses a simple 2D Fourier transform. In that case, local variations in frequency will not be taken into account, which will result in coefficients of the same frequency being matched while they might be from different parts of the image. Not only that but also coefficients are not separated by angle. The curvelet transform solves those problems while still having an explicit mathematical model.�Nevertheless, few works have been done on using neural networks with inputs and labels in the curvelet domain, whereas none are related to seismic imaging research. \citet{Saxena2015} uses the curvelet transform to extract features from fingerprint data as input to a simple feed-forward network classifier. \citet{Gupta2014} uses a similar approach to detect lung cancer CT scan images.vThis work aims to use a similar idea but to a regression problem, the LSRTM deblurring filter estimation one. However, it should be noted that this work is not a machine-learning paper since there is no learning involved. We are merely using the property neural networks of being universal function approximators \citep{Hornik1990} to approximate the inverse of the Hessian.yWe start by first describing, in Section~\ref{curveletSection}, the curvelet transform. In section~\ref{cnn}, the same is done for convolutional neural networks while also explaining the idea behind U-Net and what adaptations have to be done for it to work with LSRTM matching filters. Section~\ref{methodology} described the methodology employed, with a brief introduction to LSRTM followed by a brief description of the matching filters from the two approaches. In sections~\ref{exampleMarmousi} and~\ref{exampleSigsbee}, the filters are applied to the Marmousi and Sigsbee models, followed by a discussion session and conclusion.k5���5��